\documentclass[12pt,a4paper]{article}
\usepackage[utf8]{inputenc}
\usepackage[brazilian]{babel}
\usepackage[left=3cm,top=3cm,right=2cm,bottom=2cm]{geometry}

\usepackage{graphicx} %to use includegraphics

\usepackage{changepage} %to use adjustwidth

\usepackage{tocloft} %to use cfttoctitlefont

\usepackage{indentfirst} %to indentation

%summary configuration
\renewcommand{\cfttoctitlefont}{\hfill\Large\bfseries}
\renewcommand{\cftaftertoctitle}{\hfill\mbox{}}
\renewcommand{\cftsecleader}{\cftdotfill{\cftdotsep}}

%arial font
\renewcommand{\familydefault}{\sfdefault}

\setlength{\parindent}{1.5cm}

\cftsetindents{section}{0em}{2.5em}
\cftsetindents{subsection}{0em}{2.5em}
\cftsetindents{subsubsection}{0em}{2.5em}

\begin{document}

\begin{titlepage}
    \begin{center}
        \textbf{João Pedro Vilela Fonseca\\}
        \vspace*{\fill}
        \textbf{\Large Desenvolvimento de aplicação web para problemas de transferência de calor\\}
        \vspace*{\fill}
        \begin{figure}[h]
            \center
            \includegraphics[width=2cm]{Ufu.jpg}
        \end{figure}
        UNIVERSIDADE FEDERAL DE UBERLÂNDIA\\
        \vspace{0.3cm}
        FACULDADE DE ENGENHARIA MECÂNICA\\
        \vspace{0.3cm}
        2022
    \end{center}
\end{titlepage}

\begin{titlepage}
    \begin{center}
        \textbf{\large João Pedro Vilela Fonseca\\}
        \vspace*{\fill}
        Orientador\\
        \textbf{\large Prof. Dr. João Rodrigo Andrade\\}
        \vspace*{\fill}
        \textbf{\Large Desenvolvimento de aplicação web para problemas de transferência de calor\\}
        \vspace*{\fill}
        \begin{adjustwidth}{7cm}{0cm}
            Projeto de Conclusão de Curso apresentado ao\\Curso de Graduação em Engenharia Mecânica,\\como requisito parcial para conclusão.
        \end{adjustwidth}
        \vspace*{\fill}
        \textbf{UBERLÂNDIA - MG\\}
        \textbf{2022}
    \end{center}
\end{titlepage}

\begin{center}
    \tableofcontents
\end{center}

\newpage

\section{INTRODUÇÃO}

Os estudantes de engenharia são submetidos no clico básico ao estudo de matemática e física, de forma que possam compreender melhor as manipulações algébricas e numéricas quando iniciam os estudos específicos de engenharia. A vertente de fenômenos de transporte é carregada de aplicações matemáticas que podem prejudicar o entendimento e visualização para os alunos menos familiarizados.

Ilustrações são recursos muito utilizados por educadores e autores de livros didáticos a fim de melhorar o entendimento dos estudantes. Entretanto as ilustrações são limitadas à mídia que pertencem (geralmente livros, revistas e artigos), não podendo ser interativas ou mutáveis.

Neste projeto foi desenvolvida uma aplicação web com interface interativa, onde os estudantes e professores podem manipular os parâmetros e visualizar explicações pontuais sobre as aplicações matemáticas e físicas que envolvem alguns problemas de transferência de calor.

\end{document}
