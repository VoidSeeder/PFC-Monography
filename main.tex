\documentclass[12pt,a4paper]{article}
\usepackage[utf8]{inputenc}
\usepackage[brazilian]{babel}
\usepackage[left=3cm,top=3cm,right=2cm,bottom=2cm]{geometry}

\usepackage{graphicx} %to use includegraphics

\usepackage{changepage} %to use adjustwidth

\usepackage{tocloft} %to use cfttoctitlefont

\usepackage{indentfirst} %to indentation

\usepackage{minted}

\usepackage{multirow}

\usepackage{pgfplots}

\pgfplotsset{compat=1.18}

\usepackage{siunitx}

%summary configuration
\renewcommand{\cfttoctitlefont}{\hfill\Large\bfseries}
\renewcommand{\cftaftertoctitle}{\hfill\mbox{}}
\renewcommand{\cftsecleader}{\cftdotfill{\cftdotsep}}

%arial font
\renewcommand{\familydefault}{\sfdefault}

\setlength{\parindent}{1.5cm}

\cftsetindents{section}{0em}{2.5em}
\cftsetindents{subsection}{0em}{2.5em}
\cftsetindents{subsubsection}{0em}{2.5em}

\begin{document}

\begin{titlepage}
    \begin{center}
        \textbf{João Pedro Vilela Fonseca\\}
        \vspace*{\fill}
        \textbf{\Large Desenvolvimento de aplicação web para problemas de transferência de calor\\}
        \vspace*{\fill}
        \begin{figure}[h]
            \center
            \includegraphics[width=2cm]{Ufu.jpg}
        \end{figure}
        UNIVERSIDADE FEDERAL DE UBERLÂNDIA\\
        \vspace{0.3cm}
        FACULDADE DE ENGENHARIA MECÂNICA\\
        \vspace{0.3cm}
        2024
    \end{center}
\end{titlepage}

\begin{titlepage}
    \begin{center}
        \textbf{\large João Pedro Vilela Fonseca\\}
        \vspace*{\fill}
        Orientador\\
        \textbf{\large Prof. Dr. João Rodrigo Andrade\\}
        \vspace*{\fill}
        \textbf{\Large Desenvolvimento de aplicação web para problemas de transferência de calor\\}
        \vspace*{\fill}
        \begin{adjustwidth}{7cm}{0cm}
            Projeto de Conclusão de Curso apresentado ao\\Curso de Graduação em Engenharia Mecânica,\\como requisito parcial para conclusão.
        \end{adjustwidth}
        \vspace*{\fill}
        \textbf{UBERLÂNDIA - MG\\}
        \textbf{2024}
    \end{center}
\end{titlepage}

\begin{center}
    \tableofcontents
\end{center}

\newpage

\section{INTRODUÇÃO}

Os estudantes de engenharia são submetidos no clico básico ao estudo de matemática e física, de forma que possam compreender melhor as manipulações algébricas e numéricas quando iniciam os estudos específicos de engenharia. A vertente de fenômenos de transporte é carregada de aplicações matemáticas que podem prejudicar o entendimento e visualização para os alunos menos familiarizados.

Ilustrações são recursos muito utilizados por educadores e autores de livros didáticos a fim de melhorar o entendimento dos estudantes. Entretanto as ilustrações são limitadas à mídia que pertencem (geralmente livros, revistas e artigos), não podendo ser interativas ou mutáveis.

Neste projeto foi desenvolvida uma aplicação web com interface interativa, onde os estudantes e professores podem manipular os parâmetros e visualizar explicações pontuais sobre as aplicações matemáticas e físicas que envolvem alguns problemas de transferência de calor.

\section{Relatório}

O objetivo da primeira etapa é entender como os processadores gráficos (GPU) podem ser usados para solução de problemas de CFD (computation fluid dynamics) e qual a melhor abordagem, equilibrando o trabalho necessário para escrever o código e eficiência do mesmo. Nessa etapa serão feitas pesquisas e testes no sentido de entender como o código é feito quando se programa para GPU em Python e C++.

É esperado que GPU's tenham melhor desempenho em operações envolvendo matrizes em comparação com processadores lógicos (CPU's). Dessa forma, considerando que boa parte dos problemas de CFD resultam em muitos cálculos envolvendo matrizes, é esperado que o uso de GPU possa acelerar as soluções, que em determinadas condições envolvem muitas horas se utilizando os métodos usuais, como por exemplo o MatLab.

\subsection{O hardware}

O GPU utilizado é o modelo RTX 3070, da fabricante Nvidia, que através da plataforma CUDA (Compute Unified Device Architecture), possibilita a execução de códigos pelo método de computação paralela. Para este trabalho será usada a versão 12.7 do CUDA. Também é importante ressaltar que, conforme informações do fabricante, o chip RTX 3070 deve utilizar 8 gigabytes (Gb) de memória volátil.

O restante do hardware é composto por um CPU modelo Ryzen 7 5700X de 8 núcleos físicos e 16 virtuais, operando em overclock a 4.75 Ghz e duas placas de memória volátil de 16 Gb operando a 3200 Mhz em dual channel.

Nesse trabalho será referido como RAM e VRAM as memórias voláteis dedicadas ao CPU e ao GPU, respectivamente. Uma parcela de 16 Gb da memória RAM também pode ser usada pelo GPU, sendo essa parcela referida como memória compartilhada.

\subsection{Ambiente de execução}

Neste trabalho o sistema operacional usado será o Ubuntu 24.04.2 através da virtualização \emph{Windows Subsystem for Linux} (WSL), sendo então o kernel base especificamente na versão \emph{5.15.167.4-microsoft-standard-WSL2}.

Posteriormente, serão descritos todos os passos necessários para instalação dos drivers e programas para execução dos códigos em Pyhton e C++.

\subsection{O código}

Foi usado como ponto de partida um código em MatLab (apêndice A), desenvolvido para fins didáticos, quem tem como objetivo calcular e exibir de forma gráfica a distribuição de calor em um fluido que atravessa um canal circular.

\subsection{Python}

O código foi traduzido para Python com o objetivo de utilizar os recursos da linguagem e bibliotecas desenvolvidas pela comunidade.

O primeiro passo foi instalar o Python na versão mais recente 3.10.12, que será usada para todos os testes desse trabalho. Serão usadas as bibliotecas \emph{numpy} e \emph{cupy} para operações feitas no CPU e no GPU, respectivamente.

\subsubsection{Instalação}

O gerenciamento de pacotes de software e dependências nos sistemas de base Debian é feita nativamente através do \emph{Advanced Package Tool} (APT). Por se tratar de uma ferramenta nativa e altamente difundida pela comunidade, o APT será usado para gerenciar todos os pacotes nesse trabalho. Entretanto, qualquer outro gerenciador pode ser usado desde que respeitadas suas respectivas versões.

Antes de efetivamente iniciar a instalação é recomendada a atualização dos pacotes e dependências já instalados, evitando problemas relacionados à códigos antigos e conflitos de dependências. A verificação é feita através do comando:

\begin{minted}{bash}
    sudo apt update
\end{minted}

Caso o retorno indique que existem pacotes desatualizados o comando a seguir fará todas as atualizações necessária:

\begin{minted}{bash}
    sudo apt upgrade
\end{minted}

Por fim, é importante a execução dos dois comandos a seguir para fazer a limpeza de pacotes não mais necessários instalados e listados, respectivamente:

\begin{minted}{bash}
    sudo apt autoremove
    sudo apt autoclean
\end{minted}

Agora o sistema está preparado para a instalação do Python e do seu gerenciador de bibliotecas:

\begin{minted}{bash}
    sudo apt install python3 python3-pip python-is-python3
\end{minted}

Nota: o pacote \emph{python-is-python3} é opcional, tendo como função apenas transformar os comando \emph{python} e \emph{pip} em \emph{python3} e \emph{pip3}, respectivamente. Dessa forma pode-se omitir o caractere ''3'' ao utilizar tais comandos.

Em seguida, pode ser instalado um pacote que auxilia na identificação de drivers CUDA:

\begin{minted}{bash}
    sudo apt install nvidia-cuda-toolkit
\end{minted}

Com isso, pode-se usar o comando a seguir para verificar qual versão do CUDA está presente no sistema:

\begin{minted}{bash}
    nvidia-smi
\end{minted}

Output:

\begin{minted}[fontsize=\footnotesize]{bash}
+-----------------------------------------------------------------------------------------+
| NVIDIA-SMI 565.51.01              Driver Version: 565.90         CUDA Version: 12.7     |
|-----------------------------------------+------------------------+----------------------+
| GPU  Name                 Persistence-M | Bus-Id          Disp.A | Volatile Uncorr. ECC |
| Fan  Temp   Perf          Pwr:Usage/Cap |           Memory-Usage | GPU-Util  Compute M. |
|                                         |                        |               MIG M. |
|=========================================+========================+======================|
|   0  NVIDIA GeForce RTX 3070        On  |   00000000:05:00.0  On |                  N/A |
|  0%   45C    P8             19W /  176W |    1317MiB /   8192MiB |     16%      Default |
|                                         |                        |                  N/A |
+-----------------------------------------+------------------------+----------------------+
                                                                                         
+-----------------------------------------------------------------------------------------+
| Processes:                                                                              |
|  GPU   GI   CI        PID   Type   Process name                              GPU Memory |
|        ID   ID                                                               Usage      |
|=========================================================================================|
|    0   N/A  N/A       461      G   /Xwayland                                   N/A      |
+-----------------------------------------------------------------------------------------+
\end{minted}

Dessa forma, identifica-se que a versão instalada do Nvidia CUDA é 12.7, e portanto a versão que deve ser instalado do \emph{cupy} é \emph{cuda-12x}.

Em sequência, deve ser criado um ambiente virtual para utilização:

\begin{minted}{bash}
    python -m venv python-env
\end{minted}

Sendo \emph{python-env} o nome arbitrário para o ambiente.

O comando fará com que seja criada uma pasta \emph{python-env} no local onde foi executado.

Após criado, o ambiente virtual pode ser ativado a qualquer momento utilizando:

\begin{minted}{bash}
    source python-env/bin/activate
\end{minted}

A partir dos passos descritos acima o terminar deve explicitar que o ambiente virtual está ativo com o prefixo ''(python-env)'', da seguinte forma:

\begin{minted}{bash}
    (python-env) [usuário]@[computador]:[caminho atual]$
\end{minted}

Sendo [usuário] o nome do usuário em sessão no sistema, [computador] o nome do computador registrado, e [caminho atual] o caminho até a pasta atual.

Nota: para desativar o ambiente virtual em execução pode ser usado o seguinte comando:

\begin{minted}{bash}
    deactivate
\end{minted}

Desativar o ambiente não fará com que ele seja excluído, portanto todos os pacotes e bibliotecas instalados anteriormente persistirão ao ativá-lo novamente.

Caso seja necessário, o ambiente virtual pode ser removido apagando a pasta \emph{python-env} de forma recursiva:

\begin{minted}{bash}
    rm -rf python-env
\end{minted}

Com o ambiente virtual ativo, pode ser instalada a biblioteca cupy compatível através do gerenciador de bibliotecas do Python, juntamente com a biblioteca numpy que também será necessária nos testes:

\begin{minted}{bash}
    pip install numpy cupy-cuda12x
\end{minted}

Por fim, o código a seguir pode ser usado para verificar se o cupy foi instalado e está funcionando corretamente no ambiente de desenvolvimento:

\inputminted{python}{cfd_on_gpu/teste_cupy.py}

Considerando que o código foi escrito em um arquivo \emph{test\_cupy.py} a execução é feita através do comando a seguir:

\begin{minted}{bash}
    python ./test_cupy.py
\end{minted}

Caso tudo esteja funcional, o output esperado se parece com o seguinte:

\begin{minted}{bash}
    Array criado com CuPy: [1 2 3 4 5]
    Resultado da operação na GPU: [ 1  4  9 16 25]
    Dispositivo ativo: <CUDA Device 0>
\end{minted}

A máquina em questão possui apenas um adaptador gráfico, portanto o dispositivo é identificado na primeira posição (zero).


\subsubsection{Tradução do código}

A princípio, a tradução do código de MatLab para Python foi feita por um modelo de inteligência artificial, resultando em um código que funcionava parcialmente. Ao longo dos testes o código foi sendo modificado a fim de gerar resultados mais próximos aos do MatLab.

Os código em Python que executa as operações usando CPU e GPU podem ser consultados nos apêndices B e C.

\subsubsection{Testes iniciais}

No primeiro teste, com o número de volumes na direção y (nVolY) definido com 400, foi obtido um tempo de execução de 58,35 segundos usando CPU contra 5,72 segundos na GPU, evidenciando a agilidade extra do processador de vídeo em lidar com operações de matrizes.

A seguir, pode-se criar uma função que, dado um valor de nVolY, retorna o tempo total de execução. Será feito isso tanto para o código em CPU (apêndice D) quanto GPU (apêndice E), e então será possível comparar graficamente o quão eficiente cada um dos métodos é na execução do script proposto.

Para comparar os tempos de execução foi desenvolvida uma função (apêndice F) que executa 5 testes, variando o valor de nVolY para cada uma das funções e escrevendo os tempos de alocação e iteração em arquivos.

Nas tabelas a seguir estão expostos os tempos totais (alocação + iteração) para o código executado em CPU e GPU, respectivamente:

\begin{table}[!h]
    \caption{Tempo de execução para cada valor de nVolY em CPU}
    \label{tab:cpu-times-table}
    \vspace{0.5cm}
    \hspace{-2cm}\begin{tabular}{c|c|c|c|c|c|c|c}
        \multirow{2}{*}{nVolY} & \multicolumn{5}{c}{Tempo de execução (s)} & \multirow{2}{*}{Média} & \multirow{2}{*}{Desvio padrão} \\
            & Teste 1    & Teste 2    & Teste 3    & Teste 4    & Teste 5    &            &          \\
        50  & 0,827760   & 0,328056   & 0,300077   & 0,290090   & 0,292328   & 0,407662   & 0,235329 \\
        100 & 2,523969   & 2,501784   & 2,707829   & 3,625440   & 2,731451   & 2,818095   & 0,463151 \\
        150 & 8,148529   & 8,072223   & 7,874580   & 8,005467   & 9,167664   & 8,253693   & 0,520739 \\
        200 & 14,843985  & 14,669079  & 13,765631  & 13,835930  & 13,670256  & 14,156976  & 0,553929 \\
        250 & 21,429319  & 21,560324  & 22,564349  & 22,616004  & 22,460307  & 22,126061  & 0,580812 \\
        300 & 30,952905  & 31,095024  & 31,029816  & 30,807639  & 31,106310  & 30,998339  & 0,122910 \\
        350 & 41,511328  & 40,932770  & 40,918090  & 41,354543  & 41,058982  & 41,155143  & 0,265292 \\
        400 & 53,453211  & 54,332191  & 54,301608  & 53,609549  & 54,510705  & 54,041453  & 0,475651 \\
        450 & 67,794778  & 68,461572  & 67,716723  & 68,897778  & 68,900020  & 68,354174  & 0,575381 \\
        500 & 83,397922  & 83,206904  & 82,235203  & 84,060163  & 83,316285  & 83,243295  & 0,654615 \\
        550 & 102,563297 & 102,666959 & 108,051725 & 107,913193 & 106,444617 & 105,527958 & 2,732835 \\
        600 & 125,317569 & 125,022033 & 125,394101 & 123,311940 & 125,088841 & 124,826897 & 0,860852
    \end{tabular}
\end{table}

\begin{table}[]
    \caption{Tempo de execução para cada valor de nVolY em GPU}
    \label{tab:gpu-times-table}
    \vspace{0.5cm}
    \hspace{-1.25cm}\begin{tabular}{c|c|c|c|c|c|c|c}
        \multirow{2}{*}{nVolY} & \multicolumn{5}{c}{Tempo de execução (s)} & \multirow{2}{*}{Média} & \multirow{2}{*}{Desvio padrão} \\
            & Teste 1   & Teste 2   & Teste 3   & Teste 4   & Teste 5   &           &          \\
        50  & 0,490006  & 0,331601  & 0,491780  & 0,375072  & 0,199480  & 0,377588  & 0,121985 \\
        100 & 0,281556  & 0,280670  & 0,283004  & 0,350645  & 0,376177  & 0,314410  & 0,045641 \\
        150 & 0,531931  & 0,508864  & 1,508941  & 0,503857  & 0,498311  & 0,710381  & 0,446592 \\
        200 & 0,837360  & 0,865245  & 0,862136  & 0,821991  & 0,845416  & 0,846430  & 0,017897 \\
        250 & 1,253516  & 1,290630  & 1,304347  & 1,255839  & 1,280070  & 1,276880  & 0,022036 \\
        300 & 1,733097  & 1,797690  & 1,794362  & 1,754430  & 1,743906  & 1,764697  & 0,029601 \\
        350 & 2,334094  & 2,396163  & 2,354561  & 3,415496  & 2,470414  & 2,594146  & 0,462101 \\
        400 & 3,031880  & 3,118238  & 3,015091  & 3,117086  & 3,133413  & 3,083142  & 0,055158 \\
        450 & 6,829434  & 7,226090  & 6,012057  & 7,078231  & 7,108072  & 6,850777  & 0,490587 \\
        500 & 48,918336 & 43,488760 & 40,294673 & 42,226305 & 43,258711 & 43,637357 & 3,210093 \\
        550 & 73,000980 & 75,620902 & 74,448216 & 75,327260 & 75,167063 & 74,712884 & 1,049869 \\
        600 & 96,896951 & 96,320385 & 95,345571 & 96,579065 & 96,495460 & 96,327486 & 0,587368
    \end{tabular}
\end{table}

\newpage

Usando os dados coletados foi possível construir o gráfico de comparação:

\begin{figure}[!h]
    \centering
    \begin{tikzpicture}
        \begin{axis}[
            width=14cm,
            height=9cm,
            grid=major,
            xlabel={nVolY},
            ylabel={Tempo médio de execução (s)},
            xmin=0, xmax=650,
            ymin=0, ymax=140,
            xtick={0,50,...,600},
            ytick={0,20,...,140},
            legend pos=north west,
            title={Comparação de tempo médio: CPU vs GPU},
        ]

        % --- CPU ---
        \addplot+[
            color=red,
            mark=*,
            mark options={fill=red},
        ]
        coordinates {
            (50,0.407662)
            (100,2.818095)
            (150,8.253693)
            (200,14.156976)
            (250,22.126061)
            (300,30.998339)
            (350,41.155143)
            (400,54.041453)
            (450,68.354174)
            (500,83.243295)
            (550,105.527958)
            (600,124.826897)
        };
        \addlegendentry{CPU}

        % --- GPU ---
        \addplot+[
            color=blue,
            mark=*,
            mark options={fill=blue},
        ]
        coordinates {
            (50,0.377588)
            (100,0.314410)
            (150,0.710381)
            (200,0.846430)
            (250,1.276880)
            (300,1.764697)
            (350,2.594146)
            (400,3.083142)
            (450,6.850777)
            (500,43.637357)
            (550,74.712884)
            (600,96.327486)
        };
        \addlegendentry{GPU}

        \end{axis}
    \end{tikzpicture}
    \caption{Comparação do tempo médio de execução entre CPU e GPU}
    \label{fig:comparacao-cpu-gpu}
\end{figure}

\subsubsection{Análise dos resultados}

Nota-se através do gráfico que a GPU é mais rápida na execução das operações, entretanto fica claro que para valores do nVolY acima de 400 o tempo de execução aumenta mais do que o esperado. Isso acontece porque, o refinamento da malha aumenta o uso de memória volátil. Para valores acima de 400, a memória de 8 Gb dedicada à GPU não é suficiente para armazenar todas as matrizes e vetores usados nas operações, obrigando o sistema a usar a memória compartilhada, sendo essa parte da memória RAM do sistema.

O acesso do chip gráfico à memória compartilhada é feito através do barramento PCIe 4.0, que por sua vez é muito mais lento que o acesso direto através do memory bus (ou system bus), aumentando muito o tempo de execução quando é necessário o uso da memória compartilhada.

Mesmo no cenário que foi descrito acima, a GPU ainda atingiu tempos de execução menores que a CPU para todos os valores de nVolY testados, mostrando que é conveniente o uso de GPU para operações com matrizes.

\newpage

\appendix

\section{Código em MatLab}

\inputminted{octave}{cfd_on_gpu/aula15_exercicio_VolumesFinitos_Poiseuille_Flow_2D.m}

\newpage

\section{Código para CPU}

\inputminted{python}{cfd_on_gpu/aula15_exercicio_VolumesFinitos_Poiseuille_Flow_2D.py}

\newpage

\section{Código para GPU}

\inputminted{python}{cfd_on_gpu/aula15_exercicio_VolumesFinitos_Poiseuille_Flow_2D_GPU.py}

\newpage

\section{Função que retorna o tempo de execução em CPU}

\inputminted{python}{cfd_on_gpu/compare/cpu.py}

\newpage

\section{Função que retorna o tempo de execução em GPU}

\inputminted{python}{cfd_on_gpu/compare/gpu.py}

\newpage

\section{Função de comparação dos tempos de execução}

\inputminted{python}{cfd_on_gpu/compare/app.py}

\end{document}
